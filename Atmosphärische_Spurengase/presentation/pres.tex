\documentclass{beamer}
\usetheme{Rochester}
\usecolortheme{beaver}

\usefonttheme{professionalfonts} % using non standard fonts for beamer
\usefonttheme{serif} % default family is serif
% language
\usepackage{polyglossia}
\setmainlanguage{german}
\setotherlanguages{english}
\usepackage{microtype}

\usepackage[style=numeric,
backend=biber, sorting=none]{biblatex}		%Bibliographie

%math and theorems
\usepackage{amsmath}
\usepackage{amsthm}
\usepackage{amssymb}
\usepackage{amsopn}					%Matheoperatoren
\usepackage{mathtools}
\usepackage{mathdots}					%Punkte
\usepackage{dsfont}
\usepackage[arrow, matrix, curve]{xy}
\usepackage{nicefrac}
\usepackage{wasysym}
\usepackage{upgreek}					%Griechische Buchstaben
\usepackage{bbm}						%Mengensymbol
\usepackage{physics}					%Physiksymbole
\usepackage{relsize}						%Größenangaben
\usepackage[separate-uncertainty,
			per-mode=symbol]
			{siunitx}					%Einheiten
\usepackage{slashed}					%Dirac-Operator
%\usepackage{tikz}						%Zeichnen
\usepackage[percent]{overpic} 					%Auf Bild zeichnen
\usepackage{upgreek}					%Griechische Buchstaben


%useful packages
\usepackage{geometry}
\usepackage{xcolor}
\usepackage{graphicx}
\usepackage{float}
\usepackage{csquotes}
\usepackage{todonotes}
\usepackage{booktabs}
\usepackage{array}
\usepackage[labelfont=bf]{caption}
\usepackage{wrapfig}
\usepackage{enumitem}

\setmainfont{Linux Libertine O}
\setsansfont{Linux Biolinum O}

\usepackage{scrhack}					%Verbesserung Pakete
\usepackage{xltxtra}						%fontec

\usepackage{siunitx}
\usepackage{chemformula}

\usepackage{setspace}
\onehalfspace

\addbibresource{bibliography.bib}


\setbeamertemplate{caption}[numbered]
\setbeamertemplate{bibliography item}{-}
\setbeamertemplate{navigation symbols}{}

\renewcommand*{\bibfont}{\small}

\usepackage{algpseudocode}

\newtheorem{lem}{Lemma}
%\newtheorem{th}{Theorem}
%\newtheorem{cor}{Corollary}

\title{Atmosphärische Spurensuche}
\author{Aaron Mielke \& Thomas Ackermann}
\date{\today}


\begin{document}

\maketitle

% \tableofcontents

\begin{frame}
\frametitle{Zusammensetzung der Atmosphäre}
    \section{Theoretische Grundlagen}

\end{frame}

\begin{frame}
\frametitle{Messgrößen}
    \begin{itemize}
        \item Konzentration: Moleküle pro Volumeneinheit
        \item Mischverhältnis: Relativer Anteil von Spurengas zu Luftmenge
        \item Columnd density: $CD = \int \pho (s) \dds$
    \end{itemize}
\end{frame}

\begin{frame}
    \frametitle{DOAS}
    \textit{Differential Optical Absorption Spectroscopy}
    \begin{itemize}
        \item Verwendung: Konzentration von Spurengasen bestimmen
        \item Benutze Chrakteristische Profile von Molekülen
    \end{itemize}
\end{frame}

\begin{frame}
    \frametitle{Lambert-Beer}
    \begin{align}
    I(\lambda, L) = I_0 (\lambda) \exp (- \rho  L \sigma (\lambda) )
    \end{align}
    \begin{itemize}
        \item $L$ Länge des Mediums
        \item $\rho$ Dichte
        \item $\sigma (\lambda)$ Absorptionswirkungsquerschnitt
    \end{itemize}
\end{frame}

\begin{frame}
    \frametitle{Modifiziertes Lamber-Beer Gesetz}

\end{frame}

\begin{frame}
    \section{Versuchsdurchführung}
    \frametitle{Dunkelstrom \& Offset}
\end{frame}


\end{document}
