\documentclass[12pt, a4paper, bibliography=totoc]{scrartcl}

\usetheme{Rochester}
\usecolortheme{beaver}

\usefonttheme{professionalfonts} % using non standard fonts for beamer
\usefonttheme{serif} % default family is serif
% language
\usepackage{polyglossia}
\setmainlanguage{german}
\setotherlanguages{english}
\usepackage{microtype}

\usepackage[style=numeric,
backend=biber, sorting=none]{biblatex}		%Bibliographie

%math and theorems
\usepackage{amsmath}
\usepackage{amsthm}
\usepackage{amssymb}
\usepackage{amsopn}					%Matheoperatoren
\usepackage{mathtools}
\usepackage{mathdots}					%Punkte
\usepackage{dsfont}
\usepackage[arrow, matrix, curve]{xy}
\usepackage{nicefrac}
\usepackage{wasysym}
\usepackage{upgreek}					%Griechische Buchstaben
\usepackage{bbm}						%Mengensymbol
\usepackage{physics}					%Physiksymbole
\usepackage{relsize}						%Größenangaben
\usepackage[separate-uncertainty,
			per-mode=symbol]
			{siunitx}					%Einheiten
\usepackage{slashed}					%Dirac-Operator
%\usepackage{tikz}						%Zeichnen
\usepackage[percent]{overpic} 					%Auf Bild zeichnen
\usepackage{upgreek}					%Griechische Buchstaben


%useful packages
\usepackage{geometry}
\usepackage{xcolor}
\usepackage{graphicx}
\usepackage{float}
\usepackage{csquotes}
\usepackage{todonotes}
\usepackage{booktabs}
\usepackage{array}
\usepackage[labelfont=bf]{caption}
\usepackage{wrapfig}
\usepackage{enumitem}

\setmainfont{Linux Libertine O}
\setsansfont{Linux Biolinum O}

\usepackage{scrhack}					%Verbesserung Pakete
\usepackage{xltxtra}						%fontec

\usepackage{siunitx}
\usepackage{chemformula}

\usepackage{setspace}
\onehalfspace

\addbibresource{bibliography.bib}


\setbeamertemplate{caption}[numbered]
\setbeamertemplate{bibliography item}{-}
\setbeamertemplate{navigation symbols}{}

\renewcommand*{\bibfont}{\small}

\title{FP18 Atmospheric Trace Gases}
\author{Aaron Mielke \& Thomas Ackermann}
\date{\today}

\begin{document}

\begin{center}
	\makeatletter
	\thispagestyle{empty}
	\large{Fortgeschrittenen-Praktikum}	
	\hfill
    \large{Summer term 2019}
    \vspace{5mm}
	\rule{\textwidth}{0.2pt}
    \vfill
	\Huge\textbf{\@title} \\
	\vspace{10mm}
	\large{\@author} \\
	\normalfont
	\vfill	
	\makeatother
\end{center}

\normalsize
\newpage

\section*{Abstract}
This experiment was conducted in the scope of the advanced lab course in physics at the Heidelberg University. \\
The experiment was conducted in the week of the $8^\text{th}$ april, 2019.

\tableofcontents
\newpage
\begin{multicols}{2}
\section{Introduction}

\subsection{Composition}
Die Atmosphäre der Erde besteht aus einer Mischung verschiedener Gase.
Tabelle \ref{fig:atm_comp} zeigt eine Liste der Hauptkomponenten.
\begin{center}
\begin{tabular*}{\linewidth}{c c c}
\toprule
Gas & Symbol & Volumenanteil \\
\midrule
Stickstoff & \ch{N2} & $78.084 \%$ \\
    Sauerstoff & \ch{O2} & $20.942\%$ \\
    Argon & \ch{Ar} & 0.934 \% \\
    Kohlenstoffdioxid & \ch{CO2} & 358 \si{ppmv} \\

\bottomrule
\end{tabular*}
    \captionof{table}{Gase in der Atmosphäre} %\cite{atm_components}
    \label{fig:atm_comp}
\end{center}

\subsection{DOAS}

    \textit{Differential Optical Absorption Spectroscopy} (DOAS) wird verwendet um die Konzentration eines bestimmten Spurengases in der Atmosphäre zu bestimmen - was das Ziel dieses Experimentes ist.
Das Haupprinzip ist das Folgende: 
    Man nimmt zwei Spektren auf, eines welches auf der Erde aufgenommen wurde (also mit absorption des Sonnenlichtes durch die Atmosphäre) und eines bei dem die Atmosphäre nicht präsent ist (Von Satelit aus aufgenommen.)
Jedes Gas absorbiert bestimmte Wellenlängen des Sonnenlichts.
Wenn nun zwei Spektren, wie beschrieben, aufgenommen werden 
können das charakteristische Verhalten der Gase daran erkannt werden dass die Intensitäten an manchen stellen kleiner sind.
Wenn man nun weiß welche charakteristika zu welchen Gasen gehören kann die Dichte des Gases in der Atmosphäre bestimmt werden.

\subsubsection{Lambert-Beer Law}
Die Intensität eines Lichtestrahls (einer Elektromagnetischen Welle) nimmt durch Streuung und Absorption ab, wenn es durch Materie propagiert. 
Die Größe des verlustes kann mithilfe des Lambert-Beer Gesetzes bestimmt werden.
    Sei $I_0 (\lambda)$ die Startintensität des Lichtstrahls, dann ist die Intensität $I(\lambda, L)$ nachdem der Lichtstrahl eine Länge $L$ der Mediums überwunden hat
    \begin{align}
        I(\lambda, L) = I_0 (\lambda) \exp (- \rho L \sigma (\lambda)),
    \end{align}
    wobei $\sigma (\lambda)$ der Absorptionswirkungsquerschnitt und $\rho$ die Konzentration des Spurengases ist.

\subsection{Messgrößen}
Die Konzentration eines Spurengases ist die Zahl der Moleküle pro Volumeneinheit.
Eine andere wichtige Größe ist die mixing ratio.
Sie gibt den relativen Anteil eines Spurengases im Vergleich zur Luftmenge an und wird mit \si{[ppm]} oder \si{[ppt]} angegeben. 

\section{Versuchsdurchführung}
\subsection{Charakterisierung der Messinstrumente}
 Es gibt zwei Haupteffejte die Ungenauigkeiten bei der Messung verursachen können.
Durch Brownsche Bewegung in den Kabeln wird ein kleiner elektrischer Strom erzeugt, den man auch Dunkelstrom nennt.
Um die Größenordnung des Dunkelstromes einzuschätzen werden einminütige Messungen durchgeführt, bei denen die Kamera abgedeckt ist.
Die CCD-Kameras haben außerdem ein Offset, welcher durch viele kurze Messungen bestimmt werden kann.
Dunkelstrom und Offset müssen bei jedem folgenden Spektrum abgezogen werden.



\end{multicols}
\end{document}

