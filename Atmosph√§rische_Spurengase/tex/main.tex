\documentclass[12pt, a4paper, bibliography=totoc]{scrartcl}

% personal data
\date{\today}


% language
\usepackage{polyglossia}
\setmainlanguage{english}
\setotherlanguages{german}
\usepackage{microtype}
\usepackage{dcolumn}

\usepackage[style=numeric,
			natbib=true,
			backend=biber]{biblatex}		%Bibliographie
\usepackage[autostyle=true,
			 german=quotes]
			 {csquotes}					%Anführungszeichen
\usepackage{blindtext}


%math and theorems
\usepackage{amsmath}
\usepackage{amssymb}
\usepackage{amsopn}					%Matheoperatoren
\usepackage[amsmath,thmmarks,hyperref]{ntheorem}
\usepackage{mathtools}
\usepackage{mathdots}					%Punkte
\usepackage{dsfont}
\usepackage{upgreek}					%Griechische Buchstaben
\usepackage{bbm}						%Mengensymbol
\usepackage{physics}					%Physiksymbole
\usepackage{relsize}						%Größenangaben
\usepackage[separate-uncertainty,
			per-mode=symbol]
			{siunitx}					%Einheiten
%\usepackage{tikz}						%Zeichnen
\usepackage{upgreek}					%Griechische Buchstaben
\usepackage{enumitem}
\setlist{nolistsep}


%useful packages
%\usepackage{geometry}
\usepackage{xcolor}
\usepackage{graphicx}
\usepackage{float}
\usepackage{csquotes}
\usepackage{todonotes}
\usepackage{booktabs}
\usepackage{array}
\usepackage[labelfont=bf]{caption}
\usepackage{wrapfig}
\usepackage{enumitem}
%\usepackage{xr} % cross referencing
%\usepackage{titling}
%\usepackage{titlesec}
%\usepackage[Bjornstrup]
%			{fncychap}					%Kapitellayout


\setmainfont{Linux Libertine O}
\setsansfont{Linux Biolinum O}

\usepackage{scrhack}					%Verbesserung Pakete
\usepackage{xltxtra}						%fontec


\newcommand{\im}{\mathrm{i}}
\newcommand{\e}{\mathrm{e}}
\renewcommand{\pi}{\uppi}
\renewcommand{\epsilon}{\varepsilon}


\addbibresource{bibliography.bib}

%color settings
\definecolor{myred}{RGB}{196,19,47} 
\definecolor{myblue}{RGB}{0,139,139}


%appendix
\usepackage[toc,page]{appendix}

%killing indent
\setlength{\parindent}{0pt}
\usepackage{multicol}
\usepackage{siunitx}
\usepackage{hyperref}


\title{FP18 Atmospheric Trace Gases}
\author{Aaron Mielke \& Thomas Ackermann}
\date{\today}

\begin{document}

\begin{center}
	\makeatletter
	\thispagestyle{empty}
	\large{Fortgeschrittenen-Praktikum}	
	\hfill
    \large{Summer term 2019}
    \vspace{5mm}
	\rule{\textwidth}{0.2pt}
    \vfill
	\Huge\textbf{\@title} \\
	\vspace{10mm}
	\large{\@author} \\
	\normalfont
	\vfill	
	\makeatother
\end{center}

\normalsize
\newpage

\section*{Abstract}
This experiment was conducted in the scope of the advanced lab course in physics at the Heidelberg University. \\
The experiment was conducted in the week of the $8^\text{th}$ april, 2019.

\tableofcontents
\newpage
\begin{multicols}{2}
\section{Introduction}

\subsection{Composition}
Die Atmosphäre der Erde besteht aus einer Mischung verschiedener Gase.
Tabelle \ref{fig:atm_comp} zeigt eine Liste der Hauptkomponenten.
\begin{center}
\begin{tabular*}{\linewidth}{c c c}
\toprule
Gas & Symbol & Volumenanteil \\
\midrule
Stickstoff & \ch{N2} & $78.084 \%$ \\
    Sauerstoff & \ch{O2} & $20.942\%$ \\
    Argon & \ch{Ar} & 0.934 \% \\
    Kohlenstoffdioxid & \ch{CO2} & 358 \si{ppmv} \\

\bottomrule
\end{tabular*}
    \captionof{table}{Gase in der Atmosphäre} %\cite{atm_components}
    \label{fig:atm_comp}
\end{center}

\subsection{DOAS}

    \textit{Differential Optical Absorption Spectroscopy} (DOAS) wird verwendet um die Konzentration eines bestimmten Spurengases in der Atmosphäre zu bestimmen - was das Ziel dieses Experimentes ist.
Das Haupprinzip ist das Folgende: 
    Man nimmt zwei Spektren auf, eines welches auf der Erde aufgenommen wurde (also mit absorption des Sonnenlichtes durch die Atmosphäre) und eines bei dem die Atmosphäre nicht präsent ist (Von Satelit aus aufgenommen.)
Jedes Gas absorbiert bestimmte Wellenlängen des Sonnenlichts.
Wenn nun zwei Spektren, wie beschrieben, aufgenommen werden 
können das charakteristische Verhalten der Gase daran erkannt werden dass die Intensitäten an manchen stellen kleiner sind.
Wenn man nun weiß welche charakteristika zu welchen Gasen gehören kann die Dichte des Gases in der Atmosphäre bestimmt werden.

    \subsubsection{Lambert-Beer Law}
    Die Intensität eines Lichtestrahls (einer Elektromagnetischen Welle) nimmt ab wenn es durch Materie propagiert. 

The intensity of a beam of light (electromagnetic wave) decreases when moving through matter due to absorption and scattering.
The amount of Intensity loss can be computed via the Lambert-Beer law.
    Let $I_0 (\lambda)$ be the initial intensity of the light beam. 
    After passing through a medium with length $L$ the Intensity $I(\lambda, L)$ will be
    \begin{align}
        I(\lambda, L) = I_0 (\lambda) \exp (- \rho L \sigma (\lambda)),
    \end{align}
    where $\sigma (\lambda)$ is the absorption cross section and $\rho$ the concentration of the trace gas.


\section{Experimental Procedure}
\subsection{Characterization of the measuring instruments}
 
    There are two different main effects that can cause inaccuracy in the measurements.
Due to Brownian motion in the cables a small current can flow - the so called dark current.
    One can get an estimate for how big it is by conducting a long measurement (in our case one minute) where the camera is covered.
The other effect is the offset.
Here many short measurements were taken where the camera was also covered.
Both dark current and offset are subtracted from all proceeding spectra.



\end{multicols}
\end{document}

