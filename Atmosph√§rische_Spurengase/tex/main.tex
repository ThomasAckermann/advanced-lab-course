\documentclass[12pt, a4paper, bibliography=totoc]{scrartcl}

% personal data
\date{\today}


% language
\usepackage{polyglossia}
\setmainlanguage{english}
\setotherlanguages{german}
\usepackage{microtype}
\usepackage{dcolumn}

\usepackage[style=numeric,
			natbib=true,
			backend=biber]{biblatex}		%Bibliographie
\usepackage[autostyle=true,
			 german=quotes]
			 {csquotes}					%Anführungszeichen
\usepackage{blindtext}


%math and theorems
\usepackage{amsmath}
\usepackage{amssymb}
\usepackage{amsopn}					%Matheoperatoren
\usepackage[amsmath,thmmarks,hyperref]{ntheorem}
\usepackage{mathtools}
\usepackage{mathdots}					%Punkte
\usepackage{dsfont}
\usepackage{upgreek}					%Griechische Buchstaben
\usepackage{bbm}						%Mengensymbol
\usepackage{physics}					%Physiksymbole
\usepackage{relsize}						%Größenangaben
\usepackage[separate-uncertainty,
			per-mode=symbol]
			{siunitx}					%Einheiten
%\usepackage{tikz}						%Zeichnen
\usepackage{upgreek}					%Griechische Buchstaben
\usepackage{enumitem}
\setlist{nolistsep}


%useful packages
%\usepackage{geometry}
\usepackage{xcolor}
\usepackage{graphicx}
\usepackage{float}
\usepackage{csquotes}
\usepackage{todonotes}
\usepackage{booktabs}
\usepackage{array}
\usepackage[labelfont=bf]{caption}
\usepackage{wrapfig}
\usepackage{enumitem}
%\usepackage{xr} % cross referencing
%\usepackage{titling}
%\usepackage{titlesec}
%\usepackage[Bjornstrup]
%			{fncychap}					%Kapitellayout


\setmainfont{Linux Libertine O}
\setsansfont{Linux Biolinum O}

\usepackage{scrhack}					%Verbesserung Pakete
\usepackage{xltxtra}						%fontec


\newcommand{\im}{\mathrm{i}}
\newcommand{\e}{\mathrm{e}}
\renewcommand{\pi}{\uppi}
\renewcommand{\epsilon}{\varepsilon}


\addbibresource{bibliography.bib}

%color settings
\definecolor{myred}{RGB}{196,19,47} 
\definecolor{myblue}{RGB}{0,139,139}


%appendix
\usepackage[toc,page]{appendix}

%killing indent
\setlength{\parindent}{0pt}
\usepackage{multicol}
\usepackage{siunitx}
\usepackage{hyperref}


\title{FP18 Atmospheric Trace Gases}
\author{Aaron Mielke \& Thomas Ackermann}
\date{\today}

\begin{document}

\begin{center}
	\makeatletter
	\thispagestyle{empty}
	\large{Fortgeschrittenen-Praktikum}	
	\hfill
    \large{Summer term 2019}
    \vspace{5mm}
	\rule{\textwidth}{0.2pt}
    \vfill
	\Huge\textbf{\@title} \\
	\vspace{10mm}
	\large{\@author} \\
	\normalfont
	\vfill	
	\makeatother
\end{center}

\normalsize
\newpage

\section*{Abstract}
This experiment was conducted in the scope of the advanced lab course in physics at the Heidelberg University. \\
The experiment was conducted in the week of the $8^\text{th}$ april, 2019.

\tableofcontents
\newpage
\begin{multicols}{2}
\section{Introduction}

\subsection{Composition}
Earths atmosphere is made up of mixture of many gases.
Table \ref{fig:atm_comp} shows a list of the main components.
\begin{center}
\begin{tabular*}{\linewidth}{c c c}
\toprule
Gas & Symbol & Volume Fraction \\
\midrule
Nitrogen & \ch{N2} & $78.084 \%$ \\
    Oxygen & \ch{O2} & $20.942\%$ \\
    Argon & \ch{Ar} & 0.934 \% \\
    Carbon Dioxide & \ch{CO2} & 358 \si{ppmv} \\

\bottomrule
\end{tabular*}
    \captionof{table}{Gases in the atmosphere} %\cite{atm_components}
    \label{fig:atm_comp}
\end{center}

\subsection{DOAS}

    \textit{Differential Optical Absorption Spectroscopy} (DOAS) is used to determine the concentration of a particular trace gas in the atmosphere - which is the goal of this experiment.
    The main principle is the following: one takes two images of spectra, one which was taken on earth (so with the atmosphere) and one where the atmosphere is not present (from a satelite). 
Every gas absorbs specific wave lengths of light, so if one takes a look at the two spectr one can see that there characteristic parts of the spectrum with the atmosphere were the intensities are lower.
By knowing which characteristic part belongs to which gas one can determine the density of a specific gas.

    \subsubsection{Lambert-Beer Law}
The intensity of a beam of light (electromagnetic wave) decreases when moving through matter due to absorption and scattering.
The amount of Intensity loss can be computed via the Lambert-Beer law.
    Let $I_0 (\lambda)$ be the initial intensity of the light beam. 
    After passing through a medium with length $L$ the Intensity $I(\lambda, L)$ will be
    \begin{align}
        I(\lambda, L) = I_0 (\lambda) \exp (- \rho L \sigma (\lambda)),
    \end{align}
    where $\sigma (\lambda)$ is the absorption cross section and $\rho$ the concentration of the trace gas.


\section{Experimental Procedure}
\subsection{Characterization of the measuring instruments}
 
    There are two different main effects that can cause inaccuracy in the measurements.
Due to Brownian motion in the cables a small current can flow - the so called dark current.
    One can get an estimate for how big it is by conducting a long measurement (in our case one minute) where the camera is covered.
The other effect is the offset.
Here many short measurements were taken where the camera was also covered.
Both dark current and offset are subtracted from all proceeding spectra.



\end{multicols}
\end{document}

