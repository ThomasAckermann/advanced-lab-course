% !TeX root = ..//main.tex
\usepackage{polyglossia}					%Sprache
\usepackage{amsmath}					%Matheumgebung
\usepackage{amsopn}					%Matheoperatoren
\usepackage{amssymb}					%Mathesymbole
\usepackage{amsthm}					%Mathetheorem
\usepackage{blindtext}
\usepackage{bbm}						%Mengensymbol
\usepackage[style=numeric,
			natbib=true,
			backend=biber]{biblatex}		%Bibliographie
\usepackage[autostyle=true,
			 german=quotes]
			 {csquotes}					%Anführungszeichen
\usepackage[Bjornstrup]
			{fncychap}					%Kapitellayout
%\usepackage{graphix}					%Graphik
\usepackage{mathdots}					%Punkte
\usepackage{mathtools}					%Bugfix ams
\usepackage{microtype}					%Makrotypographie
\usepackage{physics}					%Physiksymbole
\usepackage{relsize}						%Größenangaben
\usepackage{scrhack}					%Verbesserung Pakete
\usepackage[headsepline,
			 footsepline,
			 automark]
			{scrlayer-scrpage}			%Kopfzeile
\usepackage[separate-uncertainty,
			per-mode=symbol]
			{siunitx}					%Einheiten
\usepackage{slashed}					%Dirac-Operator
%\usepackage{tikz}						%Zeichnen
\usepackage{upgreek}					%Griechische Buchstaben
\usepackage{xltxtra}						%fontec
\setmainlanguage{german}
\setmainfont{Linux Libertine O}
\setsansfont{Linux Biolinum O}

\ihead{Große Auswertung}
\ohead{\headmark}
\chead{}
\ifoot*{A. Mielke und T. Ackermann}
\automark{chapter}
\pagestyle{scrheadings}
\setcounter{tocdepth}{1}
\setcounter{secnumdepth}{1}
%\usetikzlibrary{calc}
%\usetikzlibrary{arrows.meta}
%\usetikzlibrary{patterns}

\usepackage[bookmarksopenlevel=section,
			linkcolor=blue,
			colorlinks=true,
			urlcolor=blue,
			citecolor=blue]
			{hyperref}		%Verweise, muss am Ende stehen