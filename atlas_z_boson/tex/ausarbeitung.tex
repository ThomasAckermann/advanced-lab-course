\documentclass[12pt, a4paper, bibliography=totoc]{scrreprt}


%\input{~/Latex/article/article_config.tex}
\usetheme{Rochester}
\usecolortheme{beaver}

\usefonttheme{professionalfonts} % using non standard fonts for beamer
\usefonttheme{serif} % default family is serif
% language
\usepackage{polyglossia}
\setmainlanguage{german}
\setotherlanguages{english}
\usepackage{microtype}

\usepackage[style=numeric,
backend=biber, sorting=none]{biblatex}		%Bibliographie

%math and theorems
\usepackage{amsmath}
\usepackage{amsthm}
\usepackage{amssymb}
\usepackage{amsopn}					%Matheoperatoren
\usepackage{mathtools}
\usepackage{mathdots}					%Punkte
\usepackage{dsfont}
\usepackage[arrow, matrix, curve]{xy}
\usepackage{nicefrac}
\usepackage{wasysym}
\usepackage{upgreek}					%Griechische Buchstaben
\usepackage{bbm}						%Mengensymbol
\usepackage{physics}					%Physiksymbole
\usepackage{relsize}						%Größenangaben
\usepackage[separate-uncertainty,
			per-mode=symbol]
			{siunitx}					%Einheiten
\usepackage{slashed}					%Dirac-Operator
%\usepackage{tikz}						%Zeichnen
\usepackage[percent]{overpic} 					%Auf Bild zeichnen
\usepackage{upgreek}					%Griechische Buchstaben


%useful packages
\usepackage{geometry}
\usepackage{xcolor}
\usepackage{graphicx}
\usepackage{float}
\usepackage{csquotes}
\usepackage{todonotes}
\usepackage{booktabs}
\usepackage{array}
\usepackage[labelfont=bf]{caption}
\usepackage{wrapfig}
\usepackage{enumitem}

\setmainfont{Linux Libertine O}
\setsansfont{Linux Biolinum O}

\usepackage{scrhack}					%Verbesserung Pakete
\usepackage{xltxtra}						%fontec

\usepackage{siunitx}
\usepackage{chemformula}

\usepackage{setspace}
\onehalfspace

\addbibresource{bibliography.bib}


\setbeamertemplate{caption}[numbered]
\setbeamertemplate{bibliography item}{-}
\setbeamertemplate{navigation symbols}{}

\renewcommand*{\bibfont}{\small}

\title{Studying the $Z$ Boson with the ATLAS Detector at the LHC}
\author{Aaron Mielke & Thomas Ackermann}

\begin{document}


\begin{center}
	\makeatletter
	\thispagestyle{empty}
	
	\begin{figure}[H]
	\flushright
	\includegraphics[width=0.35\textwidth]{fig/logo}
	\end{figure}
	
	\vspace{-30mm}
	
	\begin{flushleft}
	\large{\textbf{Fortgeschrittenen Praktikum} \\
		Summer term 2019} \\
	\end{flushleft}
	
	\vspace{5mm}
	
	\rule{\textwidth}{0.2pt}

	\vspace{50mm}
	\Huge\textbf{\@title} \\
	\vspace{10mm}
	\large{\@author} \\
	\normalfont
	
	\vspace{2mm}
	
	\makeatother
\end{center}

\normalsize
\newpage

\tableofcontents


\chapter*{Abstract}
This is the abstract





\chapter{Introduction}

The goal of the lab course was to analyze data from the ATLAS experiment and 
to calculate the mass of the $Z$ Boson.

\section{Drell-Yan Process}
A $Z$ Boson can be created during the so called ``Drell-Yan'' Process.
When a quark and an anti-quark collide either a virtual photon or a $Z$ Boson can be produced.

\section{ATLAS Detector}
The Detector consists of three main components: inner detector, calorimeters and the muon spectrometer.
The inner detector is mainly used to reconstruct the trajectories of electrically charged particles.



\chapter{Experimental procedure}


\nocite{*}
\appendix
\printbibliography

\end{document}

